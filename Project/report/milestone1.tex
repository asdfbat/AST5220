\documentclass[a4paper]{article}
\usepackage[utf8]{inputenc}
\usepackage[T1]{fontenc,url}
\usepackage{multicol}
\usepackage{multirow}
\usepackage{parskip}
\usepackage{lmodern}
\usepackage{microtype}
\usepackage{verbatim}
\usepackage{amsmath, amssymb}
\usepackage{tikz}
\usepackage{physics}
\usepackage{mathtools}
\usepackage{algorithm}
\usepackage{algpseudocode}
\usepackage{listings}
\usepackage{enumerate}
\usepackage{graphicx}
\usepackage{float}
\usepackage{hyperref}
\usepackage{tabularx}
\usepackage{siunitx}
\usepackage{fancyvrb}
\usepackage[makeroom]{cancel}
\usepackage[margin=2.0cm]{geometry}
\usepackage{pdfpages}
\renewcommand{\baselinestretch}{1}
\renewcommand{\exp}{e^}
\renewcommand{\b}{\boldsymbol}
\newcommand{\h}{\hat}
\newcommand{\m}{\mathbb}
\newcommand{\half}{\frac{1}{2}}
\renewcommand{\exp}{e^}
\renewcommand{\bar}{\overline}
\setlength\parindent{0pt}


\begin{document}
\title{AST5220 -- Milestone I}
\author{
    \begin{tabular}{r l}
        Jonas Gahr Sturtzel Lunde & (\texttt{jonassl})
    \end{tabular}}
% \date{}    % if commented out, the date is set to the current date

\maketitle
\vspace{2cm}

\section{Theory}
\subsection{Components of the universe}
We consider a flat, expanding universe, goverened by the $\Lambda \text{CDM}$ model. Our universe contains some densities of baryonic matter ($\rho_b$), cold dark matter ($\rho_{CDM}$), radiation ($\rho_r$), and dark energy ($\rho_\Lambda$). Let $\Omega_i = \dfrac{\rho_i}{\rho_c}$ be the \textit{relative densities} of each component. Here, $\rho_c$ is the \textit{critical density}, being the total density which would make the universe entirely flat. Since our universe is indeed flat, is follows that
\begin{equation*}
    \sum_i \rho_i = \rho_c \quad \Rightarrow \quad \sum_i \Omega_i = 1
\end{equation*}
We also define $\rho_{i,o}$ and $\Omega_{i,o}$ to represents the critical and relative densities today.

It can be shown that the density components evolve with time as $\rho_i(a) = \rho_{i,0}a^{-3(1+w_i)}$ where $w_i$ is some constant for each component. For our four compoenents, we have that
\begin{align*}
    \rho_b &= \rho_{b,0} a^{-3} \\
    \rho_\text{CDM} &= \rho_{\text{CDM},0} a^{-3} \\
    \rho_r &= \rho_{r,0} a^{-4} \\
    \rho_\Lambda &= \rho_{\Lambda,0}
\end{align*}
where $a$ is the \textit{scale factor} of the universe, describing its relative size to today, which is defined to be $a_0 = 1$.

We wish to avoid the densities where possible, and work directly with the relative densities and the scale factor. The relative densities can be rewritten to exclude $\rho_i$ the following way.
\begin{align*}
    \Omega_{i} = \frac{\rho_i}{\rho_c} = \frac{\rho_{i,0}a^{-3(1+w_i)}}{\rho_c} = \frac{\rho_{c,0}\Omega_{i,0}a^{-3(1+w_i)}}{\rho_c}
\end{align*}

Knowing that $\rho_c = \frac{3H^2}{8\pi G}$, which gives $\rho_{c,0} = \frac{3H_0^2}{8\pi G}$ we get that
\begin{align*}
    \Omega_{i} = \qty(\frac{H_0}{H})^2\Omega_{i,0}a^{-3(1+w_i)}
\end{align*}


\subsection{The Friedmann equation}
The evolution of the scale factor is goverened by the (first) Friedmann equation, which for the universe described above reads
\begin{equation}
    H(a) = \frac{\dot{a}}{a} = H_0\sqrt{\Omega_{b,0} a^{-3} + \Omega_\text{{CDM},0}a^{-3} + \Omega_{r,0}a^{-4} + \Omega_{\Lambda,0}}
\end{equation}

Since the universe takes on scales of many different orders of magnitude, the linear scale factor $a$ is not always well suited for analysis of large time spans. We introduce time(and scale) quantity
\begin{equation}
    x = \log{a} \quad \Rightarrow \quad a = \exp{x}
\end{equation}

The Friedmann equation now reads
\begin{equation}
    H(x) = H_0\sqrt{\Omega_{b,0} e^{-3x} + \Omega_\text{{CDM},0}e^{-3x} + \Omega_{r,0}e^{-4x} + \Omega_{\Lambda,0}}
\end{equation}

We also introduce the scaled Hubble parameter $\mathcal{H} = aH = \dot{a}$.


\subsection{Conformal Time}
As well as $a$, $x$, and $t$ for measuring time in the universe, we will introduce the conformal time $\eta$. $\eta$ has units of length, and represents the size of the event horizon at any given time. In other words, $\eta$ is the distance traversed by undisturbed light since the Big Bang.

[Insert definition of Conformal time]
[Why do we use length instead of time?]

\begin{align}\label{eqn:eta}
    \dv{\eta}{a} = \frac{c}{a\mathcal{H}}
\end{align}

We can rewrite the LHS as a derivative in regards to $x$ as
\begin{align*}
    \dv{\eta}{a} = \dv{\eta}{x}\dv{x}{a} = \dv{\eta}{x}\frac{1}{a}
\end{align*}

which, inserting into \ref{eqn:eta} gives us the differential equation
\begin{align*}
    \dv{\eta}{x} = \frac{c}{\mathcal{H}(x)}
\end{align*}


\section{Results}
Figure \ref{fig:Omegas} shows the distribution of relative densities over time. We observe that the very early universe is completely radiation dominated, which gradually evolves into matter domination, the turnover point being at $x=-8.7$, or $z=6000$. At $x=-3$, the universe energy content comes almost exclusively from baryons and CDM. This quickly changes into dark energy domination, at $x=-0.28$, or $z=0.33$, which is the era we're in today. We also observe that the total relative density sums perfectly to 1 at all times.

Figure \ref{fig:Eta} shows the evolution of conformal time and the Hubble parameter over time. In the two bottom plots, the Hubble parameter shows three regimes of linear behavior (although the last one is subtle in the x-dependent plot), which corresponds to exponential dependence on x, and power-law dependence on z and a. The three linear regimes are split by the matter-radiation equality, and the radiation-dark energy equality, both indicated in the plots. In the rediation dominated regime, $H \propto a^2$, during the matter dominated regine, $H \propto a^{1.5}$, and in the dark energy dominated regine, $H$ is constant.



\begin{figure}[H]
    \centering
    \includegraphics[scale=0.4]{../figs/Omegas.pdf}
    \caption{Plot showing the relative density distribution of components as function of $x=\log{a}$. Radiation, matter, and dark energy dominated eras are highlighted in their respective colors.}
    \label{fig:Omegas}
\end{figure}


\begin{figure}[H]
    \centering
    \includegraphics[scale=0.4]{../figs/Eta.pdf}
    \caption{Plots showing the comformal time (top-right), the scaled Hubble parameter (top-right), and the Hubble parameter (bottom-left) as function of $x=\log{a}$. The bottom-right panel shows the Hubble parameter as function of reshift $z$. All plots show the matter-radiation equality as a striped red line, and the matter-dark energy equality as a striped black line.}
    \label{fig:Eta}
\end{figure}



\end{document}